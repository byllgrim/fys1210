En bipolar junction transistor bruker både elektron- og hullstrøm,
og har 2 \emph{junctions} mellom ulikt dopede halvledere.
BJTer kommer i 2 typer, NPN og PNP. Vi skal se på den første av dem.

\paragraph{Collector, Base og Emittor} \mbox{} \\
En BJT består av 3 deler: collector, base og emittor.
Disse er bygget opp av tre dopede halvleder materialer, n-type og p-type.
Mellom disse regionene er pn-overganger akkurat som i dioder. \\\\
\includegraphics[width=0.5\textwidth]{./img/npn}
\\\\
Symbolet for BJTer ser slik ut for npn \\\\
\includegraphics[width=0.25\textwidth]{./img/npn-symbol} \\
Hvor den lille pilen peker mot det n-dopede materialet.
\\
Tilsvarende for pnp \\\\
\includegraphics[width=0.25\textwidth]{./img/pnp-symbol}


\paragraph{Fysisk struktur} \mbox{} \\
I virkeligheten er en BJT bygget opp lag på lag med en isolator rundt.
\\\\
\includegraphics{./img/npn-real}
