Kretsfamilier refererer til forksjellige teknikker som brukes til å
implementere logikk.
\\\\
Det finnes mange av disse: RTL, DCTL, RCTL, DTL, CTDL, HTL, ECL, PECL, LVPECL,
GTL, TTL, PMOS, NMOS, HMOS, CMOS, BiCMOS, IIl.
\\\\
Heldigvis skal vi bare se på noen få av dem.



\paragraph{Bipolare komponenter (BJT)} \mbox{} \\
Dioder-vakuumrør ble brukt i de første elektroniske datamaskinene
på 1940-tallet.
\\\\
DTL, diode-transistor logikk, ble først brukt på 50-tallet når man byttet ut
vakuumrørene med transistorer.
\\\\
ECL, emitter-coupled logikk, er raske integrerte kretser som bruker for mye
energi. De ble brukt mellom 1970-1990, men kan også bli brukt i dag.
\\\\
TTL, transistor-transistor logikk, er mye brukt i integrerte kretser.
Etter oppfinnelsen på 60-tallet er de fremdeles i bruk i dag.



\paragraph{Unipolare komponenter (FET)} \mbox{} \\
FET brukes bl.a. i NMOS og CMOS kretser.
\\\\
NMOS, ulempen med NMOS er at den bruker strøm selv når den ikke switcher.
\\\\
CMOS, den mest vanlige IC-teknologien (Integrated Circuit).
Bortsett fra lekasjestrøm bruker den kun strøm når den switcher.
\\\\
BiCMOS, kombinerer CMOS og TTL.
BJT gir fordeler for analoge deler, CMOS gir enkle logiske porter.
Ble bl.a. brukt i Pentium Pro.
