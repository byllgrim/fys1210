Vi kan implementere en NAND port med dioder og transistorer (DTL).
\\\\
\begin{tabular}{c|c|c}
  A & B & Y \\ \hline
  0 & 0 & 1 \\
  0 & 1 & 1 \\
  1 & 0 & 1 \\
  1 & 1 & 0
\end{tabular}
\\\\
\begin{circuitikz} \draw
(8,3) node[pnp] (pnp){}
(pnp.B) node[anchor=east] {}
(pnp.C) node[anchor=north] {}
(pnp.E) node[anchor=south] {}

(2,0) to[D] (0,0)
      node[label=$B$]{}
(2,0) to[short,-*] (2,3)
      to[D] (0,3)
      node[label=$A$]{}
(2,3) to[D] (4,3)
      to[D] (6,3)
      -- (pnp.B)
(6,3) to[R] (6,1)
      -- (8,1)
(2,3) -- (2,4)
      node[right]{$V_{DD}$}
      to[R] (2,6)
      -- (10,6)
      node[right,label=$V_{CC}$]{5V}
(8,6) to[R] (8,4)
      -- (pnp.E)
(8,4) -- (10,4)
      node[label=$Y$]{}
(pnp.C) -- (8,0)
      node[ground]{}
      ;
\end{circuitikz}
\\\\
Fra $V_{DD}$ kan spenningen gå til jord via A, B eller de to diodene
og transistoren. Disse er koblet i parallel, så spenningen fra $V_{DD}$ til
jord er lik uansett hvilken vei du tar.
\\\\
Når enten A eller B er null vil spenningen fra $V_{DD}$ til jord være 0.7.
Da er den 0.7 uansett hvilken vei du tar, og veien over gjennom transistoren
krever 2.1V for at transistoren skal lede, altså leder ikke transistoren.
\\\\
Når derimot både A og B er på vil all spenning fra $V_{CC}$ legge seg over de
to diodene og transistoren, og 5V er nok til at transistoren leder.
Når transistoren leder har Y klar bane til jord og er altså 0.
