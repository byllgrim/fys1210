I DTL kretser ble transistoren brukt til forsterkning, mens diodene ble
brukt for logikk.
I TTL brukes transistoren til begge deler.
\\\\
Transistoren kan ha flere emittere for å unngå \emph{dyp metning}.
Man vil kunne trekke ladning raskt ut fra base så kretsen blir mer responsiv
og kan tåle høyere hastighet.
\\\\
Man kan også bruke schottkytransistorer til å forhindre metning.
Slike kretser kalles S-TTL (Schottky TTL).
Her switcher transistoren mye raskere enn ellers, men ved høyere energibruk.
\\\\
Enter LS-TTL (Low-power Schottky TTL).
Her introduseres høyere indre motstand så det går mindre strøm i kretsen for
å redusere energibruket.
\\\\
Det finnes også Advanced Low-power Schottky (ALS) som videre forbedrer
hastighet og energiforbruk.
