\paragraph{Oppløsning} \mbox{} \\
Når man konverterer fra et analogt signal til et digitalt blir signalet
representert binært.
Oppløsningen, eller bit-dybden, er hvor mange bit vi bruker på å representere
signalet.

Med bare 2-bit får man et grovt, forstyrret signal.
Jo høyere bit depth jo mer presis blir samplingen.
8-bit gir 256 nivåer (255 hvis like mange bit skal fordeles over og under null).



\paragraph{Konverteringstid} \mbox{} \\
I de ADCene vi så på tar det tid for klokka å tikke nærmere signalets verdi.
Tiden det tar for én slik sampling kalles konverteringstid.



\paragraph{Kvantiseringsfeil} \mbox{} \\
Når vi leser av et signal fra Sample-Hold, vil det opprinelige signalet ha
forandret seg i mellomtiden.
Med dette oppstår det som kalles kvantiseringsfeil.
Med høyere oppløsning vil dette reduserer, men da vil konverteringstiden øke.
