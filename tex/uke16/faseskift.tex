Faseskift-oscillatoren produserer en sinusbølge.
Fordi faseforskyvningen avhenger av frekvens regnes den som ustabil og
er lite brukt.
Den har ingen input-signal, men trigges av en impuls.

De er sammensatt av et nettverk av RC-kretser.
For å kompansere for tapet over disse må loopgain være 1.
$$A_f = \frac{A}{1 - A\cdot \beta}$$
$$A\cdot \beta \to 1
  \implies A_f \to \infty$$

\begin{circuitikz} \draw
(8,1.5) node[op amp] (opamp) {}
(opamp.+) node[left] {}
(opamp.-) node[left] {}
(opamp.out) node[right] {}

%Resistors
(0,0) node[ground] {} -- (6,0)
(1,0) to[R] (1,2)
(3,0) to[R] (3,2)
(5,0) to[R] (5,2)
(6,0) -- (6,1)
      -- (opamp.+)

%Capasitors
(opamp.-) -- (5,2)
      to[C] (3,2)
      to[C] (1,2)
      to[C] (-1,2)
      -- (-1, 3)
      -- (9.5,3)
      -- (9.5,1.5)

%Output
(opamp.out) -- (10.5,1.5)
      node[label=$V_o$] {}
      ;
\end{circuitikz}

Opampen inverterer signalet 180 grader,
mens RC-leddene inverterer det ytterligere 180 for en gitt frekvens.
Frekvensen den oscillerer med er gitt ved:
$$f = \frac{1}{2\pi RC\sqrt{6}}$$
