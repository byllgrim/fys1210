\paragraph{} \mbox{Båndbredde} \\
Ved filtre som båndpass og båndstopp er båndbredden gitt fra grensefrekvensene.
Fra $V_{pk}$ til en reduksjon på -3dB finner man grensefrekvensene.
Båndbredden er avstanden mellom disse frekvensene.
$$BW = f_{c2} - f_{c1}$$



\paragraph{} \mbox{Q-Verdi} \\
Q-Verdien sier noe om hvor bratt et filter avtar.
En høy Q betyr brattere filter.

Q-Verdien er gitt ved den geometriske senterfrekvensen $f_0$.
$f_0$ er nesten som gjennomsnitt, men tar høyde for logaritmisk skala.
$$f_0 = \sqrt{f_{c1} \cdot f_{c2}}$$

Q-Verdien er forholdet mellom senterfrekvensen og båndbredden.
$$Q = \frac{f_0}{BW}$$



\paragraph{} \mbox{Pol} \\
Når vi lager filtre med opamper bruker vi RC kretser.
En pol er én RC krets.
Det vil bli mer tydelig i seksjonen med implementasjonseksempler.



\paragraph{} \mbox{Orden} \\
Antall poler i et filter avgjør filterets orden.
Det bestemmer også hvor fort signalet avtar.

1. Ordens filter
Har én pol.
Avtar med 20dB per dekade.

2. Ordens filter
Har to poler.
Avtar med 40dB per dekade.

3. Ordens filter
Har tre poler.
Avtar med 60dB per dekade.
osv...
