Slew rate er et mål på en krets evne til å reagere på endringer i spenning.
Spesifikasjonen til f.eks. en opamp kan garantere en viss slew rate sånn at man
kan beregne om den er brukbar i en gitt krets.
Signalet gitt på input skal kunne gjenskapes perfekt (med en hvis toleranse)
på output.

Slew rate (S) er gitt som forholdet mellom spenning/sekund.
$$S \geq \frac{dv}{dt}$$

I en forsterker må følgende tilfredsstilles
$$S \geq 2 \cdot \pi \cdot f \cdot V_{pk}$$
