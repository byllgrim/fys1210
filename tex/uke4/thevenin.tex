\subsubsection{Last-analyse}
Thevenins teorem er en regneteknikk hvor du kan betrakte noe komplisert som noe enkelt.
Det brukes som regel for å regne på forksjellig last uten å måtte regne ut hele kretsen på nytt.

\begin{circuitikz} \draw
(0,0) -- (0,4)
      node [right=3em,below=3em,anchor=west]
           {\large{\textsc{Komplisert krets}}}
      -- (6,4)
      -- (6,0)
      -- (0,0)
(6,3) to[short, -o] (7,3)
      to[voltmeter] (7,1)
      to[short, o-] (6,1)
      ;
\draw[dashed]
(7,3) -- (8,3)
      to[R, label=$R_L$] (8,1)
      -- (7,1)
      ;
\end{circuitikz}
\\
Alle topolede, lineære nettverk (krets)...
\\\\

\begin{circuitikz} \draw
(0,0) -- (0,4)
      -- (6,4)
      -- (6,0)
      -- (0,0)
(6,3) to[short, -o] (7,3)
      to[voltmeter] (7,1)
      to[short, o-] (6,1)
      -- (1,1)
      to[battery] (1,3)
      node[right=1em,below=2.5em,anchor=west]{$V_{TH}$}
      to[R, label=$R_{TH}$] (6,3)
      ;
\draw[dashed]
(7,3) -- (8,3)
      to[R, label=$R_L$] (8,1)
      -- (7,1)
      ;
\end{circuitikz}
\\
...kan erstattes med en spenningskilde $V_{TH}$ og en motstand $R_{TH}$.
\\\\
$V_{TH}$=Spenningen over polene uten last.\\
$R_{TH}$=Motstand over polene når alle spenningskilder er kortsluttet og alle strømmer brutt.



\subsubsection{Eksempel}
TODO

\subsubsection{Nortons Teorem}
TODO
