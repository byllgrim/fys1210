\subsubsection{Last-analyse}
Thevenins teorem er en regneteknikk hvor du kan betrakte noe komplisert som noe enkelt.
Det brukes som regel for å regne på forksjellig last uten å måtte regne ut hele kretsen på nytt.

\begin{circuitikz} \draw
(0,0) -- (0,4)
      node [right=3em,below=3em,anchor=west]
           {\large{\textsc{Komplisert krets}}}
      -- (6,4)
      -- (6,0)
      -- (0,0)
(6,3) to[short, -o] (7,3)
      to[voltmeter] (7,1)
      to[short, o-] (6,1)
      ;
\draw[dashed]
(7,3) -- (8,3)
      to[R, label=$R_L$] (8,1)
      -- (7,1)
      ;
\end{circuitikz}
\\
Alle topolede, lineære nettverk (krets)...
\\\\

\begin{circuitikz} \draw
(0,0) -- (0,4)
      -- (6,4)
      -- (6,0)
      -- (0,0)
(6,3) to[short, -o] (7,3)
      to[voltmeter] (7,1)
      to[short, o-] (6,1)
      -- (1,1)
      to[battery] (1,3)
      node[right=1em,below=2.5em,anchor=west]{$V_{TH}$}
      to[R, label=$R_{TH}$] (6,3)
      ;
\draw[dashed]
(7,3) -- (8,3)
      to[R, label=$R_L$] (8,1)
      -- (7,1)
      ;
\end{circuitikz}
\\
...kan erstattes med en spenningskilde $V_{TH}$ og en motstand $R_{TH}$.
\\\\
$V_{TH}$=Spenningen over polene uten last.\\
$R_{TH}$=Motstand over polene når alle spenningskilder er kortsluttet og alle strømmer brutt.



\subsubsection{Eksempel}
\begin{circuitikz} \draw{}
(0,1) -- (0,0) node[ground]{}
(0,0) to[battery, label=$V_S$] (0,4)
      to[R, label=$R_1$] (4,4)
      to[R, label=$R_2$] (4,2)
      to[R, label=$R_3$] (4,0)
      node[ground]{}
(4,2) -- (8,2)
      to[R, label=$R_L$] (8,0)
      node[ground]{}
      ;
\end{circuitikz}
\\
Denne kretsen kan skrives om til å ligne på beskrivelsen av Thevenin ovenfor.
\\

\begin{circuitikz} \draw{}
(1,1) to[battery, label=$V_S$] (1,3)
      to[R, label=$R_1$] (3,3)
      to[R, label=$R_2$] (5,3)
      to[R, label=$R_3$] (5,1)
      -- (1,1)
(5,3) to[short, -o] (7,3)
(5,1) to[short, -o] (7,1)
(0,0) -- (0,4)
      -- (6,4)
      -- (6,0)
      -- (0,0)
      ;
\draw[dashed]
(7,3) -- (8,3)
      to[R, label=$R_L$] (8,1)
      -- (7,1)
      ;
\end{circuitikz}
\\

Vi regner ut $V_{TH}$:\\
Spenning målt over polene uten last,
tilsvarer å måle spenning rundt $R_3$.\\
(Husk at $R_1$ og $R_2$ står i serie)
$$V_{TH} = V_3 = \frac{R3}{(R_1 + R_2) + R_3} \cdot V_S$$
\\

Vi regner ut $R_{TH}$:\\
Motstand over polene når spenningskilder er kortsluttet,
blir som å betrakte kretsen som en parallellkobling.
$$R_{TH} = \frac{(R_1 + R_2)R_3}{(R_1 + R_2) + R_3}$$


\subsubsection{Nortons Teorem}
Nortons teorem bygger videre på thevenins teorem.
Det sier at enhver krets, uansett hvor kompleks,
kan representeres med en strømkilde i parallell med en motstand.

\begin{circuitikz}[american currents] \draw{}
(0,0) to[I, label=$I_{Norton}$] (0,2)
      -- (2,2)
      to[R, label=$R_{Norton}$] (2,0)
      -- (0,0)
(2,2) to[short, -o] (4,2)
(2,0) to[short, -o] (4,0)
      ;
\end{circuitikz}
