\paragraph{Ideell spenningskilde} \mbox{} \\
En perfekt spenningskilde vil ha like stor spenning hele tiden,
uavhengig av hvor mye strøm den leverer.
I virkeligheten vil dette ikke være sant for en reell spenningskilde.
\\\\
Strømmen ut av en spenningskilde vil påvirkes av kildens indre motstand
(tenk thevenin).
Eksempel på indre motstand: \\
Lommelyktbatteri: 1 til 10 \SI{}{\ohm}. \\
Bilbatteri: 0.01 til 0.004 \SI{}{\ohm}.



\paragraph{Maximum power!} \mbox{} \\
Effekten $P$ fra en spenningskilde maksimaliseres når man kobler på
en lastmotstand som er \emph{lik} kildens indre motstand.
$$R_L = R_I$$
Effekt er gitt ved ligningen
$$P = \frac{U^2}{R}$$
