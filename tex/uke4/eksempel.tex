\begin{circuitikz} \draw{}
(0,1) -- (0,0) node[ground]{}
(0,0) to[battery, label=$V_S$] (0,4)
      to[R, label=$R_1$] (4,4)
      to[R, label=$R_2$] (4,2)
      to[R, label=$R_3$] (4,0)
      node[ground]{}
(4,2) -- (8,2)
      to[R, label=$R_L$] (8,0)
      node[ground]{}
      ;
\end{circuitikz}
\\
Denne kretsen kan skrives om til å ligne på beskrivelsen av Thevenin ovenfor.
\\

\begin{circuitikz} \draw{}
(1,1) to[battery, label=$V_S$] (1,3)
      to[R, label=$R_1$] (3,3)
      to[R, label=$R_2$] (5,3)
      to[R, label=$R_3$] (5,1)
      -- (1,1)
(5,3) to[short, -o] (7,3)
(5,1) to[short, -o] (7,1)
(0,0) -- (0,4)
      -- (6,4)
      -- (6,0)
      -- (0,0)
      ;
\draw[dashed]
(7,3) -- (8,3)
      to[R, label=$R_L$] (8,1)
      -- (7,1)
      ;
\end{circuitikz}
\\

Vi regner ut $V_{TH}$:\\
Spenning målt over polene uten last,
tilsvarer å måle spenning rundt $R_3$.\\
(Husk at $R_1$ og $R_2$ står i serie)
$$V_{TH} = V_3 = \frac{R3}{(R_1 + R_2) + R_3} \cdot V_S$$
\\

Vi regner ut $R_{TH}$:\\
Motstand over polene når spenningskilder er kortsluttet,
blir som å betrakte kretsen som en parallellkobling.
$$R_{TH} = \frac{(R_1 + R_2)R_3}{(R_1 + R_2) + R_3}$$
