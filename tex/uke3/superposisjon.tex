Superposisjonsprinsippet brukes til å finne verdier i kretser med mer enn én spenningskilde.
For å finne spenningen rundt en komponent ser man på bidraget fra én spenningskilde om gangen.
Når bidraget fra alle kildene er funnet, legger man det sammen for å få totalverdien.

\subsubsection{Eksempel}
$V_{S1}=\SI{15}{\volt}$,\qquad
$V_{S2}=\SI{3}{\volt}$,\qquad
$R_1=R_2=R_3=\SI{1}{\kilo\ohm}$

\begin{circuitikz} \draw
(0,0) to[battery, l=$V_{S1}$] (0,4)
      to[R, l=$R_2$] (4,4)
      -- (8,4)
      to[R, l=$R_1$] (8,0)
      -- (0,0)
(4,0) to[battery, l=$V_{S2}$] (4,2)
      to[R, l=$R_3$] (4,4)
;
\end{circuitikz}

I denne kretsen er det to spenningskilder som begge bidrar til å
skape spenning $V_1$ rundt motstanden $R_1$.
