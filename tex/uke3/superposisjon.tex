Superposisjonsprinsippet brukes til å finne verdier i kretser med mer enn én spenningskilde.
For å finne spenningen rundt en komponent ser man på bidraget fra én spenningskilde om gangen.
Når bidraget fra alle kildene er funnet, legger man det sammen for å få totalverdien.

\subsubsection{Eksempel}
\paragraph{Krets med to spenningskilder} \mbox{} \\
$V_{S1}=\SI{15}{\volt}$,\qquad
$V_{S2}=\SI{3}{\volt}$,\qquad
$R_1=R_2=R_3=\SI{1}{\kilo\ohm}$



\begin{circuitikz} \draw
(0,0) to[battery, l=$V_{S1}$] (0,4)
      to[R, l=$R_2$] (4,4)
      -- (8,4)
      to[R, l=$R_1$] (8,0)
      -- (0,0)
(4,0) to[battery, l=$V_{S2}$] (4,2)
      to[R, l=$R_3$] (4,4)
      ;
\end{circuitikz}
\\
I denne kretsen er det to spenningskilder som begge bidrar til å
skape spenning $V_1$ rundt motstanden $R_1$.



\paragraph{Bidrag fra første spenningskilde} \mbox{} \\
\begin{circuitikz} \draw
(0,0) to[battery, l=$V_{S1}$] (0,4)
      to[R, l=$R_2$] (4,4)
      -- (8,4)
      to[R, l=$R_1$] (8,0)
      -- (0,0)
(4,0) -- (4,2)
      to[R, l=$R_3$] (4,4)
      ;
\end{circuitikz}
\\
Vi later som den ene spenningskilden $V_{S2}$ ikke eksisterer
og regner ut bidraget fra $V_{S1}$.



\begin{circuitikz} \draw
(0,0) to[battery, l=$V_{S1}$] (0,4)
      to[R, l=$R_2$] (4,4)
      to[R, l=$R_{EQ}$] (4,0)
      -- (0,0)
      ;
\end{circuitikz}
\\
Motstandene $R_1$ og $R_3$ danner en parallellkobling som vi kan
betrakte som én motstand $R_{EQ}$.
\\\\
Siden $R_1$ og $R_3$ er parallellkoblet får man
$R_{EQ}$ via den \emph{inverse}.
$$\frac{1}{R_{EQ}} = \frac{1}{R_1} + \frac{1}{R_3}$$
Eller, siden det bare er to motstander, via forenklingen.
$$R_{EQ} = \frac{R_1 \cdot R_3}{R_1 + R_3} = \frac{1\cdot 1}{1+1}=\frac{1}{2}$$
%TODO: enheter
Spenningen over $R_1$ vil være den samme som over $R_3$, fordi de er parallellkoblet.
Det er den samme spenningen som over hele $R_{EQ}$.
\\
Siden vi vil finne spenningen over $R_1$ holder det da å regne ut spenningen over $R_{EQ}$.
$$V_{EQ} = V_{1(S1)} = \frac{R_{EQ}}{R_{EQ} + R_2} \cdot V_{S1}
= \frac{\frac{1}{2}}{\frac{1}{2} + 1} \cdot 15 = \SI{5}{\volt}$$

$V_{1(S1)}$ er da den delen av spenningen $V_1$ forårsaket av $V_{S1}$.



\paragraph{Bidrag fra andre spenningskilde} \mbox{} \\
\begin{circuitikz} \draw
(0,0) -- (0,4)
      to[R, l=$R_2$] (4,4)
      -- (8,4)
      to[R, l=$R_1$] (8,0)
      -- (0,0)
(4,0) to[battery, l=$V_{S2}$] (4,2)
      to[R, l=$R_3$] (4,4)
      ;
\end{circuitikz}
\\
Denne gangen later vi som $V_{S1}$ ikke eksisterer.



\begin{circuitikz} \draw
(0,0) to[battery, l=$V_{S2}$] (0,4)
      to[R, l=$R_3$] (4,4)
      -- (8,4)
      to[R, l=$R_2$] (8,0)
      -- (0,0)
(4,0) to[R, l=$R_1$] (4,4)
      ;
\end{circuitikz}
\\
Tegnet på en annen måte ser vi at $R_1$ og $R_2$ også
danner en parallellkobling. Den kan vi betrakte som $R_{FQ}$
og regne ut på samme måte.
\\
Totalmotstanden til $R_{FQ}$ gis på samme måte som ista.
$$R_{FQ} = \frac{R_1 \cdot R_2}{R_1 + R_2} = \frac{1}{2}$$
\\
Spenningen over $R_{FQ}$ er lik spenningen over $R_1$
som er lik spenningen over $R_2$.
$$V_{FQ} = V_{1(S2) = \frac{R_{FQ}}{R_{FQ} + R_3} \cdot V_{S2}}
= \frac{\frac{1}{2}}{\frac{1}{2} + 1} \cdot 3 = \SI{1}{\volt}$$



\paragraph{Total spenning!} \mbox{} \\
Nå som vi har regnet ut begge bidragene $V_{1(S1)}$ og $V_{1(S2)}$
kan vi legge dem sammen og få den totale spenningen $V_1$.
$$V_1 = V_{1(S1)} + V_{1(S2)} = 5 + 1 = \SI{6}{\volt}$$
