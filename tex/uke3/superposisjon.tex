Superposisjonsprinsippet brukes til å finne verdier i kretser med mer enn én spenningskilde.
For å finne spenningen rundt en komponent ser man på bidraget fra én spenningskilde om gangen.
Når bidraget fra alle kildene er funnet, legger man det sammen for å få totalverdien.

\subsubsection{Eksempel}
$V_{S1}=\SI{15}{\volt}$,\qquad
$V_{S2}=\SI{3}{\volt}$,\qquad
$R_1=R_2=R_3=\SI{1}{\kilo\ohm}$



\begin{circuitikz} \draw
(0,0) to[battery, l=$V_{S1}$] (0,4)
      to[R, l=$R_2$] (4,4)
      -- (8,4)
      to[R, l=$R_1$] (8,0)
      -- (0,0)
(4,0) to[battery, l=$V_{S2}$] (4,2)
      to[R, l=$R_3$] (4,4)
      ;
\end{circuitikz}
\\
I denne kretsen er det to spenningskilder som begge bidrar til å
skape spenning $V_1$ rundt motstanden $R_1$.



\begin{circuitikz} \draw
(0,0) to[battery, l=$V_{S1}$] (0,4)
      to[R, l=$R_2$] (4,4)
      -- (8,4)
      to[R, l=$R_1$] (8,0)
      -- (0,0)
(4,0) -- (4,2)
      to[R, l=$R_3$] (4,4)
      ;
\end{circuitikz}
\\
Vi later som den ene spenningskilden $V_{S2}$ ikke eksisterer
og regner ut bidraget fra $V_{S1}$.



\begin{circuitikz} \draw
(0,0) to[battery, l=$V_{S1}$] (0,4)
      to[R, l=$R_2$] (4,4)
      to[R, l=$R_{EQ}$] (4,0)
      -- (0,0)
      ;
\end{circuitikz}
\\
Motstandene $R_1$ og $R_3$ danner en parallellkobling som vi kan
betrakte som én motstand $R_{EQ}$.
Siden $R_1$ og $R_3$ er parallellkoblet får man
$R_3$ via den \emph{inverse}.
$$\frac{1}{R_{EQ}} = \frac{1}{R_1} + \frac{1}{R_3}$$
Eller, siden det bare er to motstander, via forenklingen.
$$R_{EQ} = \frac{R_1 \cdot R_3}{R_1 + R_3} = \frac{1\cdot 1}{1+1}=\frac{1}{2}$$
%TODO: enheter
