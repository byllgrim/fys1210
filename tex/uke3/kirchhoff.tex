\subsubsection{Kirchhoffs lov om strømmer}
TODO



\subsubsection{Kirchhoffs lov om spenninger}
TODO



\subsubsection{Spenningsdeler}
Vi ser på tilfellet med to motstander seriekoblet til et batteri.
\\
\begin{circuitikz} \draw
(0,0) to[battery, l=$V_{batteri}$] (0,4)
      -- (4,4)
      to[R, l=$R_2$] (4,2)
      to[R, l=$R_1$] (4,0)
      -- (0,0)
      ;
\end{circuitikz}
\\
Hva er spenningen $V_1$ over motstanden $R_1$?
$$V_1 = \frac{R_1}{R_1 + R_2} \cdot V_{batteri}$$

Du kan tenke på det som dette:
\\
Hvor stor del av kaka tar $R_1$?
sin rettferdige andel: $\frac{R_1}{R_1 + R_2}$
\\
Hvor mye kake er det egentlig? $V_{batteri}$
