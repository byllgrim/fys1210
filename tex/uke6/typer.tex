\paragraph{Zener} \mbox{} \\
Zener-dioder viker i \emph{Reverse Breakdow} området i diodens virkeområde.
Det vil si, når spenningen er revers i forhold til diodens retning.
\\
De brukes som bl.a. spenningsregulator fordi
stor forandring i strøm fører til liten forandring i spenning.
Da får man en referansespenning man kan designe etter.

\paragraph{Schottky} \mbox{} \\
Schottky-dioder er nyttige når man vil forhindre energitap.
De har et lavere spenningsfall enn vanlige dioder.
Hvor en vanlig diode har mellom 0.6-0.7 volt er schottky mellom 0.15-0.45 V.

\paragraph{Varicap} \mbox{} \\
Varicap-diode, variable capacitance diode,
brukes som en variabel kondensator.
Kan brukes i radioaparater for å stille inn ønsket frekvens.
