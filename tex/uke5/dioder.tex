En diode har to poler, en anode og en katode,
og kan lede elektrisk strøm kun fra anoden til katoden.
Snur vi dioden så vil den fungere som en sperre.
Derfor sier man ofte at en diode har en lederetning og en sperreretning.

En diode er I stand til å stoppe ganske mye volt,
men om det er koblet for å lede strøm så tåler den ikke så mye.
\\
\includegraphics[width=0.5\textwidth]{./img/diode-symbol}
\\

Ved hjelp av en diode kan vi sende strømmen i den retningen vi vil,
og beskytte mot situasjoner som overslag.

En diode som vanligvis består av silisium
vil ikke lede strøm før den blir påtrykt med en spenning på ca. 0,7 V.
Dette har konsekvenser for resten av komponentene i kretsen,
for om vi sender 12V inn over dioden,
så vil det være igjen ca. 11,3 V for resten av komponente.
Det vil si at vi må huske på at
en diode stjeler litt av spenningen fra kretsen.
