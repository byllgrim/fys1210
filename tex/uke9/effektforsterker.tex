\paragraph{Med avkoblet emitter} \mbox{} \\
\begin{circuitikz} \draw
(4,3) node[npn] (npn) {}
      (npn.base) node[anchor=east] {}
      (npn.collector) node[anchor=south] {}
      (npn.emitter) node[anchor=north] {}

(0,3) node[label=$in$] {}
      to[C, l=$C_1$] (2,3)
      to[R, l=$R_1$] (2,6)
      -- (4,6)
      to[short, -o, l=$V_{CC}$] (4,7)
(4,6) to[R, l=$R_C$] (npn.collector)

(2,3) -- (npn.base)
(npn.emitter) to[R, l=$R_E$] (4,0)
      -- (1,0)
      node[ground] {}
(2,3) to[R, l=$R_2$] (2,0)

(4,2) -- (6,2)
      to[C, l=$C_3$] (6,0)
      -- (4,0)

(4,4) -- (5,4)
      to[C, l=$C_2$] (7,4)
      to[short, -o, l=$out$] (8,4)
      ;
\end{circuitikz}
\\
I denne kretsen er forsterkningen gitt ved
$$A_V = -g_m \cdot R_C$$
\\
Eventuelt, hvis du har en last på output
$$A_V = -gm \cdot (R_C || R_L)$$
\\
Hvor vi \emph{husker} at
$$g_m = \frac{I_C}{V_T}$$



\paragraph{Uten avkoblet emitter} \mbox{} \\
Uten avkoblet emitter kan du ta vekk kondensatorene på høyre side av kretsen.
Da blir forsterkningen (uten last)
$$A_V = -\frac{R_C}{R_E}$$



\paragraph{Småsignaler} \mbox{} \\
Kretsen ovenfor sett ifra småsignalmodellen gir
\\
\begin{circuitikz} \draw
(1,0) to[R, l=$R_{B1}$] (1,4)
(3,0) to[R, l=$R_{B1}$] (3,4)
(6,0) to[R, l=$R_E$] (6,2)
      -- (5,2)
      to[R, l=$r_\pi$] (5,4)
      node[label=$B$] {}
      to[short, -o] (0,4)
      node[label=$V_{inn}$] {}
(6,2) -- (7,2)
      to[american controlled current source] (7,4)
      node[label=$C$] {}
      to[short, -o] (9,4)
      node[label=$V_{ut}$] {}
(8,4) to[R, l=$R_C$] (8,0)
(9,0) to[short, o-o] (0,0)
      ;
\end{circuitikz}
\\
Motstanden som signalet ser inn mot kretsen er
$$R_{inn} = R_{B1} || R_{B2} || r_{inn}$$
\\
Hvor $r_{inn}$ er motstanden etter de 2 parallell-motstandene.
$$r_{inn} = r_\pi + (\beta + 1)R_E$$
