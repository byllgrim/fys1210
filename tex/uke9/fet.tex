FET, field effekt transistor, er en type transistor.
En FET kalles en \emph{unipolar} komponent fordi de har én type bærere.
Ladningstransporten skjer ved majoritetsbærere.

Fordeler:
\begin{itemize}
\item Spenningsregulert: Kan anses som en spenningskontrollert strømkilde.
\item Veldig stor inngangsmotstand. Som gjør den energieffektiv.
\end{itemize}

Ulemper:
\begin{itemize}
\item Lav transkonduktans. Som gir liten forsterkning.
\end{itemize}



\subsubsection{To typer}
Field effekt transistorer kommer i 2 typer
\\
\paragraph{JFET: Junction Field Effekt Transistor} \mbox{} \\
JFET er den første typen FET som ble laget.
De deles igjen i to typer, n-channel og p-channel, avhengig av hva slags doping
som er brukt på den innerste halvlederen.
\\
\paragraph{MOSFET: Metall Oksyd Semiconductor FET} \mbox{} \\
MOSFET er en nyere teknologi enn JFET.
Kommer også som n-channel eller p-channel, men kommer i tillegg som en av:\\
E-MOSFET - Enhancement mode MOSFET (er på med tilstrekkelig spenning på gate) \\
D-MOSFET - Depletion mode MOSFET (er på uten spenning på gate)



\subsubsection{JFET}
TODO

\subsubsection{MOSFET}
TODO

\subsubsection{CMOS}
TODO
