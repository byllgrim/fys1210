Oppgaven skal gi en innføring i digitale kretser.
\\\\
Vi skal se hvordan digitale kretser er bygget opp av komponenter.
Og hvordan egenskapene til disse bestemmer kretsens ytelse.
Det er også en øvelse i bruk av oscilloskop og signalgenerator,
med vekt på oscilloskopets \emph{cursor}-funksjon.
\\\\
Utgangspunktet er i en diode-transistor krets (DTL - diode transistor logikk).
Vi ser på en NAND-krets og skal se på de logiske og elektriske egenskapene.
Hva skjer når man legger til en Schottky-diode mellom collector og base på
en switch-transistor?
\\\\
Vi skal også undersøke egenskapene til en low-power-schottky NAND.
