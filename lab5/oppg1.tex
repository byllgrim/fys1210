Kretsen vår er en NAND.
Det er to input, A og B, og en output Y.
Spenningen avgjør om signalet regnes som 1 eller 0.
\\\\
Et potmeter gjør at vi kan justere spenningen på A. \\
Input B er alltid 1, men en knapp setter den til 0. \\
En lysdiode, aktivert av strap S2, indikerer om output er 1 eller 0.

\subsection{Potmeter}
Hvor går spenningsgrensen mellom 1 og 0 i logikken?
\\
Hvis B alltid er på, og lampa av når begge er på, skrur vi A opp til lampa
er av.
Da kom vi opp til ~1.1 volt.
Det betyr at grense mellom 1 og 0 (i port-logikken) er ca 1.1V.

\subsection{Oppg 1}
Prøv forskjellige kombinasjonene av A og B og verifiser funksjonstabellen.
\\\\
\begin{tabular}{l|c|r}
  A & B & Y \\ \hline
  0 & 0 & 1 \\ \hline
  0 & 1 & 1 \\ \hline
  1 & 0 & 1 \\ \hline
  1 & 1 & 0 \\
\end{tabular}
\\\\
Strap S2 er på så vi kan observere om output er 1 eller 0.
Deretter vrir vi potmeteret for å justere A mellom 0 og 5 volt.
\begin{itemize}
\item Vi skrur A=0 og trykker ned B=0.
      Dioden lyser! Akkurat som forventet.
\item Fortsatt er A=0 og slipper opp B=1.
      Dioden lyser! Akkurat som forventet.
\item Vrir opp A=1 og trykker ned B=0.
      Dioden lyser! Akkurat som forventet.
\item Fortsatt er A=1 og slipper opp B=1.
      Det kommer en tid i enhver diodes liv hvor lyset tar slutt.
\end{itemize}
